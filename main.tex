\documentclass[letter]{article}
\usepackage{sectsty}
\sectionfont{\fontsize{8}{2}\selectfont}
\font\myfont=cmr12 at 12pt
\title{{\myfont Flavor wave model homework problems}}
\usepackage{natbib}
\usepackage{graphicx}
\usepackage{titlesec}
\usepackage{bm}
\usepackage{amssymb}
\usepackage{braket}
\usepackage{mathtools}
\usepackage{amsmath}
\usepackage{mathrsfs}
\usepackage{ulem}
\setcounter{MaxMatrixCols}{20}
\usepackage{siunitx}
\usepackage{mathabx}
\usepackage{ulem}
\usepackage{textgreek}
%\usepackage[T1]{fontenc}
%\usepackage[type1]{libertine}
%\usepackage{newtxmath}
\let\oldemptyset\emptyset
\let\emptyset\varnothing
\titlelabel{\thetitle.\quad}
\usepackage{geometry}
\geometry{left=1cm,right=0.5cm,top=0.5cm,bottom=2cm}
\newcommand{\stkout}[1]{\ifmmode\text{\sout{\ensuremath{#1}}}\else\sout{#1}\fi}
\newcommand{\e}{\mathrm{e}}
\newcommand{\ii}{\mathrm{i}}
\newcommand{\dd}{\mathrm{d}}
\usepackage{upgreek}
%\makeatletter
%\re@DeclareMathSymbol{\aalpha}{\mathord}{lettersA}{11}
%\makeatother
\begin{document}
\maketitle
\indent A. In this part, we will develop a method solving the ground state without diagonalizing the Hamiltonian.\\
\indent (1).  Assume a particle can be discribed by spin only, with $S=1$. and the Hamiltonian is 
$$ H= S_z.
$$
Write down the matrix form for $H$, the eigenstates of $H$ and their energies. We will write down the ground state to be $\left|a_0\right>$\\
\indent (2). Assume a initial state to be $\left|\psi_0\right>=\frac{1}{\sqrt{3}}(1,1,1)$. Let $\varepsilon=10^{-2}$. Calculate 
$$\left|\phi_1\right>=1-\varepsilon H \left|\psi_0\right>$$
$$\left|\psi_1\right>=\frac{\left|\phi_1\right>}{\left<\phi_1|\phi_1\right>}$$
Look at the $\left|\psi_1\right>$, and show that  there are more weights of ground state in $\left|\psi_1\right>$ thatn $\left|\psi_0\right>$.\\
\indent (3). Define 
$$\left|\phi_{n+1}\right>=1-\varepsilon H \left|\psi_n\right>$$
$$\left|\psi_{n+1}\right>=\frac{\left|\phi_{n+1}\right>}{\left<\phi_{n+1}|\phi_{n+1}\right>}.$$
Draw (i)$\left|\left<a_0|\psi_n\right>\right|^2$ as a function of n. and 
(ii) $\left<\psi_n|H|\psi_n\right>$ as a function of n. Also try with differenct positive values of $\varepsilon$ and see how the value will change the speed of converging.\\
\indent (4). repeat (2) and (3) with $\left|\psi_0\right>=(0,1,0)$ 
\indent (5). For a general Hamiltonian
$$ H = \sum_{i=0}^{m-1} E_{i} \left|a_i\right>\left<a_i\right|
$$
\end{document}
